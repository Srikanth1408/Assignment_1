\documentclass{article}
\twocolumn
\usepackage[utf8]{inputenc}
\usepackage{geometry}
\usepackage{amsmath}
\usepackage{amsthm}
\usepackage{mathtools}
\usepackage{commath}

\title{\textbf{\Huge Assignment 1}}
\author{\large H.N Srikanth - SM21Mtech12012}

\date{August 2021}


\begin{document}

\providecommand{\mbf}{\mathbf}


\newcommand{\myvec}[1]{\ensuremath{\begin{pmatrix}#1\end{pmatrix}}}
\let\vec\mathbf


\maketitle

\section*{Chapter II, Examples II}
\textbf{Q22 (iii)}
\textbf{Find the conditions that the four points}
\myvec{x_1\\y_1}, \myvec{x_2\\y_2},
\myvec{x_3\\y_3}, \myvec{x_4\\y_4}
\textbf{ may be the vertices of a rhombus.}\\

\textbf{Solution :}
\vspace{0.2cm}
 The given points are


\begin{align*}
\vec{A} = \myvec{x_1\\y_1}, \vec{B} =\myvec{x_2\\y_2},
\vec{C} =\myvec{x_3\\y_3}, \vec{D} =\myvec{x_4\\y_4},
\end{align*}


Condition for the given four points be the vertices of a rhombus are ;-\\
1) If distances of all the four sides are equal\\
2) If opposite sides are parallel and\\
2) If diagonals are perpendicular bisectors.

\vspace{0.2cm}

Let us consider two vectors say,
\begin{align*}
\vec{U} = \myvec{u_1\\u_2}, \vec{V} =\myvec{v_1\\v_2}
\end{align*} 


then distance can be calculated using norm of a vector, i.e., 
$$\norm{ \vec{U} - \vec{V}} = \sqrt{(v_1-u_1)^2+(v_2-u_2)^2}$$
Here, $$ D_1=\norm{ \vec{A} - \vec{B}} =\sqrt{(x_2-x_1)^2+(y_2-y_1)^2}$$
$$ D_2=\norm{ \vec{B} - \vec{C}} =\sqrt{(x_3-x_2)^2+(y_3-y_2)^2}$$
$$ D_3=\norm{ \vec{C} - \vec{D}} =\sqrt{(x_4-x_3)^2+(y_4-y_3)^2}$$
$$ D_4=\norm{ \vec{D} - \vec{A}} =\sqrt{(x_1-x_4)^2+(y_1-y_4)^2}$$

\vspace{20cm}



if $$( A-C )^T . ( B-D ) = 0 $$\\
implies that AC and BD are perpendicular to each other.\\ 

and say E , F be mid points joining the lines AC and BD. Now if\\
AE = EC \\
BF = FD \\ 
implies that AC and BD bisect each other futher E and F be a same point.\\

The above two conditions prove that AC and BD are perpendicular bisectors.
\vspace{5mm} %5mm vertical space

Now if \\ 1) $D_1=D_2=D_3=D_4\\ 
2) \pm(A-B) = \pm \hspace{1mm} (D-C) ,\hspace{2mm} \pm(B-C)= \pm(A-D)\\
  (implies\hspace{2mm} AB // DC\hspace{2mm} and \hspace{2mm}BC // AD)\\
\text{3) AC and BD are perpendicular bisectors}

\vspace{5mm} %5mm vertical space
\textbf{Then, we can say that the given points are the vertices of a rhombus.}

\end{document}