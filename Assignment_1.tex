\documentclass{article}
\twocolumn
\usepackage[utf8]{inputenc}
\usepackage{geometry}
\usepackage{amsmath}
\usepackage{amsthm}
\usepackage{mathtools}
\usepackage{commath}

\title{\textbf{\Huge Assignment 1}}
\author{\large H.N Srikanth - SM21Mtech12012}

\date{August 2021}


\begin{document}

\providecommand{\mbf}{\mathbf}


\newcommand{\myvec}[1]{\ensuremath{\begin{pmatrix}#1\end{pmatrix}}}
\let\vec\mathbf


\maketitle

\section*{Chapter II, Examples II}
\textbf{Q22 (iii)}
\textbf{Find the conditions that the four points}
\myvec{x_1\\y_1}, \myvec{x_2\\y_2},
\myvec{x_3\\y_3}, \myvec{x_4\\y_4}
\textbf{ may be the vertices of a rhombus.}\\

\textbf{Solution :}
\vspace{0.2cm}
 The given points are


\begin{align*}
\vec{A} = \myvec{x_1\\y_1}, \vec{B} =\myvec{x_2\\y_2},
\vec{C} =\myvec{x_3\\y_3}, \vec{D} =\myvec{x_4\\y_4},
\end{align*}


Conditions for the given four points to be the vertices of a rhombus are ;-\\
1) If opposite sides are parallel and\\
2) If diagonals are perpendicular .

\vspace{0.2cm}


if $$(\vec{A}-\vec{B} ) = k.(\vec{D}-\vec{C} )$$
 $$(\vec{B}-\vec{C} )=k.(\vec{A}-\vec{D} )$$

 \vspace{0.2cm}
 shows AB // DC and BC // AD $$



if $$(\vec{A}-\vec{C} )^ \top. ( \vec{B}-\vec{D} ) = 0 $$\\
implies that AC and BD are perpendicular to each other.\\ 





\vspace{5mm} %5mm vertical space
\textbf{Now if given four points satisfy the above conditions then, we can say that the given points are the vertices of a rhombus.}

 
 \vspace{20cm}
\textbf{Numerical Example :}
 \vspace{0.2cm}

Examine whether the given points A (2,-3) and B (6,5) and C (-2,1) and D (-6,-7) forms a rhombus.

 \textbf{Sol:}
 The given points are
 \begin{align*}
\vec{A} = \myvec{2\\-3}, \vec{B} =\myvec{6\\5},
\vec{C} =\myvec{-2\\1}, \vec{D} =\myvec{-6\\-7},
\end{align*}
$$(\vec{A}-\vec{B} ) = \myvec{-4\\-8}, (\vec{D}-\vec{C} ) = \myvec{4\\8}$$

$$(\vec{B}-\vec{C} ) = \myvec{8\\4}, (\vec{A}-\vec{D} ) = \myvec{8\\4}$$
$$(\vec{A}-\vec{B} ) = -(\vec{D}-\vec{C} )$$
$$(\vec{B}-\vec{C} ) = (\vec{A}-\vec{D} )$$
This shows AB//DC and BC//AD
$$(\vec{A}-\vec{C} )^\top = \begin{matrix}
(4 & -4 )
\end{matrix},(\vec{B}-\vec{D} ) = \myvec{12\\12}$$

$$(\vec{A}-\vec{C} )^ \top. ( \vec{B}-\vec{D} ) = 48-48 = 0

This shows AC and BD are perpendicular.
 \vspace{0.2cm}

\textbf{Given points A,B,C,D satisfy both the conditions hence they form a Rhombus}





 


\end{document}